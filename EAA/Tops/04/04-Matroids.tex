\subtop{Matroide und der Greedy Algorithmus}
\vspace*{-0.75\baselineskip}
\subsection{Matroid}
\begin{description}
	\item[Unabhängigkeitssystem:] endliche Menge $X$ und eine Menge $\I$ von Teilmengen von $X$ für die gilt:
		\begin{enumerate}
			\item $\emptyset \in \I$
			\item falls $I_2 \in \I$ und $I_1 \subseteq I_2$ dann gilt $I_1 \in \I$
		\end{enumerate}
	\item[Austauscheigenschaft:] falls $I_1,I_2 \in \I$ und $|I_1|<|I_2|$ dann gibt es ein $x\in I_2\setminus I_1$ sodass $I_1 \cup \{x\} \in \I$
	\item[Matroid:] Unabhängigkeitssystem mit Austauscheigenschaft
		\example{Matroid}{\ \\\up
			\begin{itemize}
				\item ein endlicher Vektorraum mit der Menge an unabhängigen Teilmengen
				\item Kantenmenge eines Graphs zusammen mit der Menge von kreisfreien spannenden Teilgraphen
			\end{itemize}
		}
\end{description}
\topbreak
\up\up
\begin{description}
	\item[Kreis eines Unabhängigkeitssystems:] kleinste Teilmenge von $X$, die nicht in $\I$ ist
	\item[Basis eines Unabhängigkeitssystems:] größtes Element aus $\I$; alle Basen eines Matroids haben die gleiche Größe (Folgerung aus Austauscheigenschaft)
\end{description}
\subsection{Greedy Algorithmus}
\begin{description}
	\item[Voraussetzungen:]\ \\\up
		\begin{enumerate}
			\item Unabhängigkeitssystem $(X,\I)$ mit Gewichtsfunktion $w : X \rightarrow \mathbb{R}$
			\item $w(X') = \sum\limits_{x\in X'} w(x)$ ist das Gewicht einer Teilmenge $X' \subseteq X$
		\end{enumerate}
	\item[Nutzen:] berechnet Basis mit kleinstem Gewicht
\end{description}
Wenn $M=(X,\I)$ ein Matroid ist, so berechnet der Greedy-Algorithmus die kleinste Basis im Bezug auf die Gewichtsfunktion.\\
\missProof\\\ \\