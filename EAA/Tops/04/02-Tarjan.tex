\subtop{Tarjan's Kantenfärbungs-Methode}
\begin{itemize}
	\item \textbf{Farbeninvariante:} Es gibt einen MST, der alle blauen und keine rote Kante enthält.
	\item eine Kante $e=\{v,w\}\in E$ \textbf{kreuzt} einen \emph{Schnitt}, falls $v \in S \subsetneq V$ und $w \in V \setminus S$
	\item ein \textbf{einfacher Kreis} ist ein verbundener (Teil-)Graph mit $\forall v \in V : deg(v)=2$
	\item wenn $T$ ein Spannbaum ist, so gibt es für jeden Schnitt in $G$ eine Kante, die diesen Schnitt kreuzt, sowie es in jedem Kreis eine Kante gibt, die nicht in $T$ ist
\end{itemize}
\begin{description}
	\item[Blaue Regel:] Auswählen eines Schnittes, den keine blaue Kante kreuzt $\rightarrow$ färbe Kante mit dem kleinsten Gewicht blau
	\item[Rote Regel:] Auswählen eines einfachen Kreises, der keine rote Kante enthält $\rightarrow$ färbe die Kante mit dem größten Gewicht rot
\end{description}
Dieser Algorithmus wird solange angewendet, bis keine Regel mehr angewendet werden kann.
\topbreak
\up\ \\Tarjan's Kantenfärbungsalgorithmus färbt alle Kanten richtig.
\up\Proof
Am Anfang ist keine Kante gefärbt. Da der Graph verbunden ist, gibt es auch einen MST. Nach dem $k$-ten Schritt gibt es einen MST $T$ mit allen blauen und keinen roten Kanten. Jetzt gibt es zwei Fälle:
\begin{description}
	\item[Anwendung der blauen Regel:] Falls der Algorithmus eine Kante $e \in T$ färbt, ist alles ok. Sonst gibt es eine Kante $e'$ auf dem Schnitt $C=(S,V\setminus S)$ die nicht blau gefärbt ist und zu $T$ gehört (sie kann nicht rot sein, sonst wäre sie nicht im Baum $T$). Dann färben wir die Kante $e$ blau. Da immer die Kante mit dem kleinsten Gewicht genommen wird, gilt $w(T')\leq w(T)$.
	\item[Anwendung der roten Regel:] Äquivalent zur blauen Regel mit einem Kreis $C$ sowie der Folgerung, dass $w(e) \geq w(e')$ und $w(T')\leq w(T)$.
\end{description}
Um zu zeigen, dass der Algorithmus auch alle Kanten färbt müssen wir folgende zwei Fälle zeigen:
\begin{description}
	\item[$e \in T$:] Betrachten der beiden Komponenten, die durch den Schnitt $C$ durch $e$ entstehen: keine blaue Kante geht über $C$, somit können wir $e$ blau färben.
	\item[$e \notin T$:] Betrachten den Kreis $C$ (der einzigartige Pfad von $v$ nach $w$, wobei $e=\{v,w\}$), dann gibt es keine rote Kante auf $C$ und wir können die rote Regel anwenden.
\end{description}\ \\