\subtop{Master Methode (Master Theorem)}
	\begin{itemize}
		\item[a)] generelle Lösung für Rekursionsformeln der Form\\
		$T(n) = a \cdot T(\dfrac{n}{b})+f(n)$
		\item[b)] $a,b \geq 1$ sind Konstanten
		\item[c)] $f: \mathbb{N}\rightarrow \mathbb{R}_{\geq 0}$
		\item[d)] erste Annahme: $n=b^k \left(\frac{n}{b^k} = 1 \Leftrightarrow k = \log_b n\right)$:\\
			\scalebox{0.75}{\usetikzlibrary{positioning}
\begin{tikzpicture}[every node/.style={draw,circle},node distance=0.7cm]
	\node (a) at (0,0) {$n$}
		child {node(1)[xshift=-2cm] {$\frac{n}{b}$}
			child {node(b)[xshift=-1cm] {$\frac{n}{b^2}$}
				child {node(4)[draw=none] {$\vdots$}
					child {node(c) {$\frac{n}{b^k}$}}
					child {node {$\frac{n}{b^k}$}}
				}
				child {node[draw=none] {$\vdots$}}
			}
			child {node[xshift=1cm] {$\frac{n}{b^2}$}
				child {node[draw=none] {$\vdots$}}
				child {node[draw=none] {$\vdots$}}
			}
		}
		child {node(2)[xshift=2cm] {$\frac{n}{b}$}
			child {node[xshift=-1cm] {$\frac{n}{b^2}$}
				child {node[draw=none] {$\vdots$}}
				child {node[draw=none] {$\vdots$}}
			}
			child {node[xshift=1cm] {$\frac{n}{b^2}$}
				child {node[draw=none] {$\vdots$}}
				child {node(5)[draw=none] {$\vdots$}
					child {node {$\frac{n}{b^k}$}}
					child {node {$\frac{n}{b^k}$}}
				}
			}
		};
\node(3)[draw=none] at(0,-1.5){$\dots$ $a$ many $\dots$};
\foreach \x in {1,2,4,5}{
	\node[draw=none,below=of \x] {$\dots$};
}
\node[draw=none,below=of 3,yshift=-1.9cm] {$\dots$};
\foreach \x/\y in {1/0,2/-1.5,3/-3,4/-6}{
	\node[draw=none] (x\x) at (-8,\y) {\x:};
}
\foreach \x/\y in {1/a,2/1,3/b,4/c}{
	\draw[dashed](x\x)--(\y);
}

\end{tikzpicture}}
			\begin{description}
				\item[1.] $f(n)$
				\item[2.] $f(n) + a\cdot f\left( \dfrac{n}{b}\right)$
				\item[3.] $f(n) + a\cdot f\left( \dfrac{n}{b}\right) + a^2\cdot f\left( \dfrac{n}{b^2}\right)$
				\item[4.] $f(n) + a\cdot f\left( \dfrac{n}{b}\right) + a^2\cdot f\left( \dfrac{n}{b^2}\right) + c_0\cdot a^k$ (wobei $k \approx \log_b n$)
				\item[Endsumme:] $c_0\cdot \underbrace{a^{\log_b n}}_{n^{\log_b a}} + \sum\limits_{i=0}^{\log_b n -1} a^i \cdot f\left(\dfrac{n}{b^i}\right)$
			\end{description}
		\item[e)] somit gilt in Rekursionsschritt $i$: zusätzlicher Aufwand von $a^i f\left(\dfrac{n}{b^i}\right)$
		\item[f)] falls in Rekursionstiefe $k$ der Wert $\dfrac{n}{b^k}$ klein genug ist, kann er durch die Konstante $c_0$ ersetzt werden
	\end{itemize}
	\subsection{Laufzeit}
		$T(n)=c_0\cdot \underbrace{a^{\log_b n}}_{n^{\log_b a}} + \sum\limits_{i=0}^{\log_b n -1} a^i \cdot f\left(\dfrac{n}{b^i}\right)$
	\topbreak
	\vspace*{-2\baselineskip}\subsection{Laufzeitbestimmung mit dem Master Theorem}
		$a \geq 1, b> 1, \epsilon>0, f : \mathbb{N} \rightarrow \mathbb{R}_{\geq 0}$, sowie $T(n)= a \cdot T(\frac{n}{b}) + f(n)$\hfill ($\frac{n}{b}$ ist entweder $\floorFrac{n}{b}$ oder $\ceilFrac{n}{b}$)
		\begin{description}
			\item[Fall 1:]
				\begin{description}
					\item[Voraussetzung: ]$f(n)\in \BigO(n^{\log_b a - \epsilon})$ für beliebiges $\epsilon>0$
					\item[Folgerung: ]$T(n) \in \BigO(n^{\log_b a})$
					\item[Beispiel: ]$T(n) = 8 T\left( \dfrac{n}{2}\right) + 1000n^2$\\
					$\Rightarrow a = 8, b = 2, f(n) = 1000n^2,   \log_b a = \log_2 8 = 3\\
					\Rightarrow 1000n^2 \in \BigO\left(n^{3 - \epsilon} \right)$
				\end{description}
			\item[Fall 2:] 
				\begin{description}
					\item[Voraussetzung: ]$f(n) \in \Theta\left( n^{\log_b a} \right)$
					\item[Folgerung: ]$T(n) \in \Theta\left( n^{\log_b a} \log n\right)$
					\item[Beispiel: ]$T(n) = 2 T\left(\dfrac{n}{2}\right) + 10n$\\
					$\Rightarrow a = 2, b = 2, f(n) = 10n, \log_b a = \log_2 2 = 1\\
					\Rightarrow 10n \in \Theta\left( n^{1} \right)$
				\end{description}
			\item[Fall 3:] 
				\begin{description}
					\item[Voraussetzung: ]$f(n) \in \Omega\left( n^{\log_b a + \epsilon} \right)$ für ein $\epsilon>0$ und falls die Regularitätsbedingung gilt (ein $c$ mit $0 < c < 1$: $	a\cdot f\left(\dfrac{n}{b}\right) \leq c\cdot f(n)$)
					\item[Folgerung: ]$T(n) \in \Theta(f(n))$
					\item[Beispiel: ]$T(n) = 2 T\left(\dfrac{n}{2}\right) + n^2$\\
					$\Rightarrow a = 2, b = 2, f(n) = n^2, \log_b a = \log_2 2 = 1\\
					\Rightarrow n^2 \in \Omega\left( n^{1 + \varepsilon} \right)$\\
					Regularitätsbedingung: $2 \left(\half{n}\right)^2 \leq c\cdot n^2 \Leftrightarrow \half{1} n^2 \leq cn^2\\
					\Rightarrow T(n) \in \Theta(n^2)$
				\end{description}
		\end{description}
	