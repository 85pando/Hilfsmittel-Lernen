\subsection{Algorithmus von Chan}
\begin{itemize}[itemsep=-1pt]
	\item kombiniert Graham's Scan und Jarvis' march
	\item berechnet die konvexe Hülle in $\BigO(n\log h)$, mit $h=\#$ Knoten auf der konvexen Hülle
	\item es kann gezeigt werden, dass die Berechnung für die Entscheidung, ob eine Menge mit $n$ Punkten $h$ Elemente auf der konvexen Hülle hat, in $\Omega(n\log h)$ geschehen kann (\textit{Kirckpatrick und Seidel})
	\item der Algorithmus ist optimal \textit{output-sensitive}
\end{itemize}
\topbreak
\up\up
\begin{itemize}
	\item berechnet die konvexe Hülle einer Punktemenge durch anwenden des Jarvis' march und die folgenden beiden Tricks:
		\begin{description}
			\item[Trick 1:]\ \\\vspace*{-1.5\baselineskip}
				\begin{enumerate}[itemsep=0pt]
					\item vorläufige Annahme: $h$ ist bekannt
					\item Vorverarbeitung: Aufteilen von $Q$ in $\ceilFrac{n}{h}$ Teilmengen der Größe $\leq h$
					\item Benutzen von Graham's Scan um alle konvexen Hüllen der Teilmengen zu berechnen\\
					$\Rightarrow \BigO(\frac{n}{h}h\log h)$
					\item Anwenden von Jarvis' march
					\item in jedem Schritt von Jarvis' march muss der nächste Knoten auf der konvexen Hülle aus den Knoten der konvexen Hüllen der Teilmengen kommen
					\item der Kandidat aus jeder Teilmenge kann in $\BigO(\log h)$ mithilfe der binären Suche gefunden werden
					\item Laufzeit von Jarvis' march liegt in $\BigO(h\frac{n}{h}\log h)$
				\end{enumerate}
			\item[Trick 2:] \ \\\vspace*{-1.5\baselineskip}
				\begin{enumerate}
					\item wenn man $h$ nicht im Voraus weiß, arbeitet der Algorithmus in verschiedenen Phasen
					\item in jeder Phase wird ein Parameter $m$ (anstelle von $h$) verwendet, der die Menge von Punkten in Teilmengen der maximalen Größe $m$ teilt
					\item im Schritt des Jarvis' march werden nicht notwendigerweise alle Knoten der konvexen Hülle berechnet, aber bis zu $m+1$ Knoten der konvexen Hülle
					\item Start des nächsten Schrittes
					\item Parameter $m$ wird erhöht durch Quadrieren von $2$ bis hinzu $h\leq m<h^2$
				\end{enumerate}
		\end{description}
	\item detailliertere Version des Algorithmus:
		\begin{itemize}[itemsep=-2pt]
			\item Start mit $m=2$, Aufrechterhaltung der Bedingung $m < h^2$
			\item folgende Schritte werden in jeder Phase durchgeführt:
				\begin{enumerate}[itemsep=-2pt]
					\item Aufteilen der Menge $Q$ von Punkten in $r=\ceilFrac{n}{m}$ Mengen $Q_1,\dots,Q_r$ mit jeweils höchstens $m$ Elementen
					\item für alle $i=1,\dots,r$ wird Graham's Scan angewendet, um in $\BigO(m\log m)\subseteq \BigO(m\log h)$ die konvexe Hülle von $Q_i$ zu berechnen und die geordnete Sequenz ihrer Knoten in einem Array $H(Q_i)$ zu speichern\\
					$\Rightarrow$ insgesamt ergibt sich eine Laufzeit von $\BigO(rm\log m)=\BigO(n\log m)$ pro Phase
					\item mit Jarvis' march werden in $\BigO(mr\log m)=\BigO(n\log m)$ höchstens $m+1$ Knoten der konvexen Hülle von $Q$ wie folgt berechnet:
						\begin{itemize}
							\item[$\rhd$] Beginn mit $k=1$, dem am weitesten links liegende Punkt der untersten Punkte $p_1$ von $Q$ und $H(Q)=\langle p_1\rangle$
							\item[$\rhd$] Durchführen, solange $k\leq m$ und $H(Q)$ das Folgende nicht abgeschlossen hat:
								\begin{enumerate}
									\item nächster Knoten $p_{k+1}$ von $H(Q)$ ist das Minimum im Bezug auf $<_{p_k}$\\
									da $p_{k+1}$ der nächste Knoten aus einer der $H(Q_i)$, kann $p_{k+1}$ in zwei Schritten berechnet werden:
										\begin{enumerate}
											\item für $i=1,\dots, r$ benutzt man binäre Suche, um in $\BigO(\log m)$ den Punkt $q_i$ zu berechnen, der das Minimum im Bezug auf $<_{p_k}$ist, aus allen Knoten aus $H(Q_i)$\\
											$\Rightarrow$ pro Phase braucht dieser Schritt insgesamt $\BigO(mr\log m)=\BigO(n\log m)$
											\item Berechnen des minimalen Punktes $p_{k+1}$ in Bezug auf die Ordnung $<_{p_k}$ aus allen Punkten $q_1,\dots,q_r$\\
											$\Rightarrow$ pro Phase braucht dieser Schritt insgesamt $\BigO(mr)=\BigO(n)$
										\end{enumerate}
									\item falls $p_{k+1}=p_1 \Rightarrow H(Q)$ ist vollständig
									\item sonst wird $p_{k+1}$ zu $H(Q)$ hinzugefügt und $k$ um eins erhöht 
								\end{enumerate}
						\end{itemize}
					\item falls $H(Q)$ noch nicht vollständig ist, wird $m$ auf $m^2$ erhöht und eine neue Phase begonnen
				\end{enumerate}
		\end{itemize}
\end{itemize}
\topbreak
\up\up
\begin{itemize}
	\item \textbf{Laufzeit:}
		\begin{itemize}
			\item für jede Phase: $\BigO(n\log m)$ (mit $m=2^{2^i}$)
			\item Algorithmus stoppt, sobald $m\geq h$
			\item $k$ wird so gewählt, dass $2^{2^{k-1}}<h\leq 2^{2^k}$ gilt
			\item Gesamtlaufzeit: $\BigO(\sum\limits_{i=0}^{k}n\log 2^{2^i})$ mit $\sum\limits_{i=0}^{k}n\log 2^{2^i} = n\sum\limits_{i=0}^{k} 2^i = n(2^{k+1}-1)<4n2^{k-1}<4n\log h$
		\end{itemize}
\end{itemize}