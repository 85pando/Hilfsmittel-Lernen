\subsection{Anwendung: Minimale Spannbäume}
\begin{itemize}
	\item benutzen eines beliebigen MST-Algorithmus (mit $\Theta(n^2)$ Kanten)
	\vspace*{-.5\baselineskip}\item brauchen Graphen mit weniger Kanten aber allen MSTs: \textbf{\dg/Triangulation} mit den Knoten $P$
	\vspace*{-.5\baselineskip}\item der \dg~ist der Dual-Graph des Voronoi-Diagrammes von $P$
	\vspace*{-.5\baselineskip}\item ein MST von $G=(P, \aoverb{V}{2})$ ist ein Teilgraph des \dg:\vspace*{-1.5\baselineskip}
		\Proof\vspace*{-.5\baselineskip}
			\begin{itemize}
				\item $\{p_1,p_2\}$ ist eine Kante eines MST in $G$
				\item betrachten des kleinsten Kreises, der $p_1,p_2$ enthält (der Kreis mit Durchmesser $\oben{p_1p_2}$)
				\item wenn es einen Punkt $p$ auf diesem Kreis gibt gilt:
				\begin{center}
					$d(p_1,p_2)>d(p_1,p)\text{ und } d(p_1,p_2)>d(p_2,p)$
				\end{center}
				\item $p_1,p,p_2$ ist ein Kreis in $G$ und eine Kante, die das größte Gewicht in einem Kreis hat, ist niemals Teil eines MST (rote Regel)\\
				$\Rightarrow$ es gibt keinen Punkt $p$
				\item somit gibt es nur einen Kreis mit $p_1,p_2$ ohne einen Punkt aus $P$ innerhalb oder auf dem Kreis
				\item da die Voronoi-Zellen von $p_1,p_2$ durch eine Kante getrennt sind\\
				$\Rightarrow \{p_1,p_2\}$ ist eine Kante des \dg
			\end{itemize}
	\vspace*{-.5\baselineskip}\item ein EMST mit $n$ Punkten in der Ebene, kann in $\BigO(n\log n)$ berechnet werden\up
		\Proof\vspace*{-.5\baselineskip}
			\begin{itemize}
				\item Voronoi-Diagramm kann in $\BigO(n\log n)$ berechnet werden
				\item der \dg~kann in $\BigO(n)$ berechnet werden
				\item der \dg~hat $n$ Knoten und höchstens $3n-6$ Kanten
				\item mit Kruskal's Algorithmus kann der MST in $\BigO(n\log n)$ berechnet werden 
			\end{itemize}
\end{itemize}