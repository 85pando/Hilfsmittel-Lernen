\begin{TOP}{Fibonacci-Heaps}
\up\up\begin{itemize}
	\item Wald aus (Min-)Heaps
	\item Element mit dem kleinsten Schlüssel ist die Wurzel jedes Baumes
	\item Min-Zeiger auf kleinste Wurzel
	\item Wurzeln sind in einer \textit{Root}-Liste gespeichert
	\item Knotennamen sind die Schlüssel der Elemente
\end{itemize}
\usetikzlibrary{positioning,arrows}

\begin{tikzpicture}[every edge/.style={draw,->,>=stealth}]
\newcommand{\phan}[1]{\node(#1)[draw,circle] at(#1){\phantom{1}};}
\node (8-1) at (0,0) {8};
\phan{8-1}
\node (2) [right=of 8-1,xshift=1.25cm] {2};
\phan{2}
\node (3) [right=of 2,xshift=0cm] {3};
\phan{3}
\node (5) [right=of 3,xshift=.5cm] {5};
\phan{5}
\foreach \name/\position/\of/\shift/\out/\s in {12-1/below left/8-1/0.5/12/-0.1,
12-2/below/8-1/0/12/0,
9/below right/8-1/0/9/-0.1,
16/below/12-2/0/16/0,
10/below left/9/1/10/-0.1,
12-3/below right/9/-1/12/-0.1,
13/below/12-3/0/13/0,
4/below/3/0/4/0,
6-1/below left/5/1/6/-0.1,
6-2/below right/5/-1/6/-0.1,
8-2/below/6-2/0/8/0
}{
	\node (\name) [\position=of \of,xshift=\shift cm,yshift=\s cm,yshift=0.5cm] {\out};
	\phan{\name}
}
\foreach \x/\y in {12-1/8-1,12-2/8-1,9/8-1,16/12-2,10/9,12-3/9,13/12-3,4/3,6-1/5,6-2/5,8-2/6-2}{
	\draw(\x)edge(\y);
}
\draw[dashed](-1.5,0)to(8-1)to(2)to(3)to(5)to(7.5,0);
\node(rl) at(8.5,0){\textit{Root}-Liste};
%\foreach \x/\y in {12-1/12-2,12-2/9,10/12-3}{
%	\draw[<->,dashed,>=stealth](\x)to(\y);
%}

\node (min)[above=of 2,yshift=-0.5cm] {\textit{min}};
\draw(min)edge(2);
\end{tikzpicture}
\begin{description}
	\item[Operationen:]\ \\\up
	\begin{description}
		\item[\insert(item $x$, key $k$):] Einfügen des Elementes $x$ mit Schlüssel $k$ als neue Wurzel in der \textit{Root}-Liste, eventuelles Updaten des Min-Zeigers
		\item[\exMin:]\ \\\up
			\begin{enumerate}
				\item alle Kinder des Minimums werden in die \textit{Root}-Liste eingefügt
				\item das Minimum wird entfernt
				\item Funktion \cons~wird auf der \textit{Root}Liste aufgerufen
			\end{enumerate}
		\item[\decKey(item $x$, key $k$):] \ \\\up
			\begin{enumerate}
				\item $k$ wird der neue Schlüssel von $x$
				\item falls $k<key[parent]$ wird der Teilbaum $T_x$ mit Wurzel $x$ abgeschnitten und die $x$ in die \textit{Root}-Liste eingefügt
				\item Update des Min-Zeigers
				\item falls der Elternknoten von $x$ schon ein Kind verloren hat, werden alle übrig gebliebenen Teilbäume (deren Elternknoten $parent[x]$ ist) in die \textit{Root}-Liste eingefügt (\textbf{cascading cut})
			\end{enumerate}
		\item[\cons:] solange es zwei Wurzeln gibt mit der gleichen Anzahl an Kindern, wird der Baum mit dem größeren Schlüssel an den Baum mit dem kleineren Schlüssel angehängt, hiernach muss der Min-Zeiger erneuert werden\\
		\algo{arg1}{arg2} %TODO Algorithmus
	\end{description}
\end{description}
	\loadTop{05/01-DatenFelder}
	\loadTop{05/02-Laufzeit}
\end{TOP}