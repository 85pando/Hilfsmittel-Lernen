\begin{TOP}{Amortisierte Analyse}
	Ein Algorithmus kann aus mehreren Operationsabfolgen bestehen. Hier kann man eine obere  Grenze der Worst-Case-Laufzeit bestimmen, indem man die Worst-Case-Laufzeit einer Operation nimmt und sie mit der Anzahl an Operationen multipliziert. Die wirkliche Worst-Case-Laufzeit kann jedoch besser sein.
	\example{MultiPop}{\ \\\vspace*{-\baselineskip}
		\begin{description}
			\item[Push($element$):] $element$ wird dem Stack hinzugefügt
			\item[MultiPop($k$):] $k$ Elemente werden vom Stack geholt (wenn weniger als $k$ Elemente auf dem Stack sind, werden alle geholt)
		\end{description}
	}
	\loadTop{02/01-Accounting}
	\loadTop{02/02-PotentialFunction}
\end{TOP}

