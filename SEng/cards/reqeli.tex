\card{Interviews/ strukturierte Befragungsaufgaben}{
	\begin{enumerate}
		\item \textbf{context-of-system:} wieso wir dieses System entwickeln, wer die Benutzer sind, kritische Funktionalität, Anforderungen
		\item \textbf{open-ended questions:} produziert eine große Menge an Informationen, falls vorher nicht viel bekannt war
		\item \textbf{close-ended questions:} spezifische Fragen
		\item \textbf{rephrase questions:} stellt sicher, dass die Fragen richtig verstanden wurden/Inkonsistenz/Unklarheiten
	\end{enumerate}
}

\card{Lernen von existierenden Systemen}{
	\begin{enumerate}
		\item Analyse von Benutzeranleitungen
		\item benutzen/spielen mit existierenden Systemen
		\item Marktanalyse (konkurrierende Systeme, Marktforschung, welche Eigenschaften eingebunden werden sollen)
		\item Reverse Engineering
	\end{enumerate}
}

\card{Unsicherheit von Anforderungen}{
	Wenn der Kunde nicht weiß, was er wirklich braucht/will $\Rightarrow$ Prototypen. Der Prototyp sollte früh gebaut werden (nur wenn dies passiert, ist es schneller einen Prototyp zu entwickeln, als das System zu bauen) und zur Anforderungsbestimmung benutzt werden. Prototyping ist auch ein Teil von verschiedenen Lebenszyklen und Prozessmodellen (Spiral-Modell).
}

\card{Anforderungen sammeln}{
	\begin{enumerate}
		\item erzeuge Ideen (frei von Kritik/Urteil) $\Rightarrow$ so viele Ideen wie möglich
		\item diskutieren, überarbeiten, organisieren der Ideen
		\item bewerten, priorisieren
	\end{enumerate}
}

\card{\textsc{Fast} (Facilitated Application Specification Technique/Erleichterte Anwendungsspezifikationstechnik)}{
	\begin{enumerate}
		\item überwinden des wir/die-Denkens (Entwickler, Benutzer, Kunde)
		\item Team-orientiert (Zusammenarbeit)
		\item Richtlinien:
			\begin{compactenum}
				\item Teilnehmen am ganzen Treffen/Meeting ist ein Muss
				\item Teilnehmer sind gleichberechtigt
				\item Vorbereitung ist wichtiger als das Meeting
				\item Vor-Meeting-Dokumente sind nur "`vorgeschlagen"'
				\item externer Standort wird bevorzugt
				\item setze eine Agenda und behalte sie bei
				\item keine technischen Details
			\end{compactenum}
	\end{enumerate}
}

\card{ANsätze zu FAST}{
	\begin{enumerate}
		\item JAD (IBM), die Methode "`Performance Resources"'
	\end{enumerate}
}

%\card{\textsc{Jad} (Joint Application Design, 1977)}{
%	\begin{compactenum}
%		\item Technik, um alle Teilnehmer dazu zubewegen ihre Zustimmung zu den Softwareanforderungen und dem Design zu geben
%		\item 20\% Kostenreduktion im Lebenszyklus, 15\% der Funktionalität wird weniger vermisst
%		\item JAD/Plan $\Rightarrow$ Softwareanforderungserhebung
%		\item JAD/Design $\Rightarrow$ Softwaredesign
%		\item Anpassung (Vorbereiten der Aufgaben für die Sitzung)
%		\item Sitzung (maßgeschneidert, vereinfachte Treffen (Entwickler und Benutzer))
%		\item Wrap-Up (Ergebnisse der Sitzung abschließen)
%	\end{compactenum}
% %}
%\card{\textsc{Jad} Hauptziele}{
%	\begin{enumerate}
%		\item Gruppendynamik (die Fähigkeiten der Individuen verbessern)
%		\item Kommunikation und Verstehen (Benutzen von visuellen Hilfen)
%		\item rationaler, organisierter Prozess (Wiederholbarkeit)
%		\item standardisierte Dokumentationen (Benutzen von Standardformen)
%	\end{enumerate}
%}
%\card{\textsc{Jad} Teilnehmer}{
%	\begin{enumerate}
%		\item Sitzungsleiter (verwaltet die Sitzung, Management-/ Kommunikationsfähigkeiten, Kompetenz von Problembereichen)
%		\item Analytiker (Sitzungsoutput, technisches Verständnis der Anforderungen)
%		\item ausführender Auftraggeber (high-level strategischer Einblick in das System, Entscheidungen auf Führungsebene (Ressourcen, \dots))
%		\item Repräsentant der Benutzer ("`User"')
%		\item Vertreter der Informationssystemeo (Machbarkeitsstudie)
%		\item Spezialist (Wissen über die Applikationsdomain)
%	\end{enumerate}
%}
%\card{\textsc{Jad} Anpassungsphase}{
%	\begin{compactenum}
%		\item Orientierung (Arbeitgeber billigt das Projekt, Sitzungsleiter und Analytiker gewinnen ein Verständnis für das System/die Umgebung)
%		\item Organisation des Teams (Teilnehmer auswählen, im Voraus Gedanken über Fragen machen)
%		\item individualisieren des Prozesses (wie viel Zeit/Ressourcen, anpassen der Dokumente)
%		\item Vorbereiten der Materialien (Folien, Präsenationen, White-Boards, Flip Charts, Agenda)
%	\end{compactenum}
%}
%\card{\textsc{Jad} Sitzungsphase}{
%	\begin{compactenum}
%		\item Orientierung (Willkommensgruß, Überblick)
%		\item definieren von high-level Anforderungen (Ziele, Vorteile, Strategien/ künftige Überlegungen, Einschränkungen und Annahmen, Sicherheit/Prüfung/Kontrolle) $\Rightarrow$ aufgezeichnet von Analytiker, Diskussion (verfeinern, beurteilen von Anforderungen)
%		\item definieren des Umfangs des Systems  (organisieren und priorisieren von Anforderungen, Umfängen, etwas, das Objektivität erfüllt, aber nicht zu kostspielig oder zu komplex ist)
%		\item vorbereiten von JAD/Design (schätzen der Ressourcen, identifizieren der Teilnehmer, Treffen planen)
%		\item Dokumentfragen und Überlegungen (Probleme die die Anforderungen beeinflussen $\Rightarrow$ Zuweisung zu einer Person zur Auflösung)
%		\item Abschluss (Auflistung von Informationen und Entscheidungen,\\Äußern von verbliebenen Bedenken, beinhaltet Verantwortliche und ihren Aufgabenbereich und die überwiegende Meinung)
%	\end{compactenum}
%}
%\card{\textsc{Jad} Wrap-Up-Phase}{
%	\begin{enumerate}
%		\item Transformation des Sitzungsprotokolls in ein formales Planungsdokument (Analytiker)
%		\item bearbeiten des JAD/Design
%		\item Bewertung des JAD/Design (alle Teilnehmer, Änderungen\\sind möglich)
%		\item Zustimmung des Geldgebers
%	\end{enumerate}
%}